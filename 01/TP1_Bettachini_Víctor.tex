\documentclass[12pt, spanish, a4paper, landscape]{article}
% Versión 1.er cuat 2017 Víctor Bettachini < victorb@gmx.net >

\usepackage{babel}
\addto\shorthandsspanish{\spanishdeactivate{~<>}}
\usepackage[utf8]{inputenc}

\usepackage{float}

%% unidades, isótopos, notación física
\usepackage[locale=FR, per-mode=fraction, separate-uncertainty=true]{siunitx}
\sisetup{detect-all}
%\DeclareSIUnit\torr{torr}
%\DeclareSIUnit\atm{atm}
%\usepackage{isotope} % $\isotope[A][Z]{X}\to\isotope[A-4][Z-2]{Y}+\isotope[4][2]{\alpha}$
\usepackage[arrowdel]{physics}

\usepackage{amsmath}
\usepackage{amstext}
\usepackage{amssymb}

\usepackage{booktabs} % table rules

\usepackage{graphicx}
\graphicspath{{./graphs/}}

\usepackage{tikz}
\usetikzlibrary{decorations.pathmorphing}
\usetikzlibrary{patterns}
% \input{DimLinesTikz}

\usepackage[margin=1.3cm,nohead]{geometry}

\usepackage{lastpage}
\usepackage{fancyhdr}
\pagestyle{fancyplain}
\fancyhead{}
% \fancyfoot{{\tiny \textcopyright V. A. Bettachini}}
\fancyfoot{{Víctor A. Bettachini}}
% \fancyfoot{{\tiny \textcopyright Universidad Provincial de Ezeiza}}
\fancyfoot[C]{ Visualización de datos | TP 1 }
\fancyfoot[R]{ \today}
% \fancyfoot[R]{Pág. \thepage/\pageref{LastPage}}
% \fancyfoot[RO, LE]{Pág. \thepage/\pageref{LastPage}}
\renewcommand{\headrulewidth}{0pt}
\renewcommand{\footrulewidth}{0pt}

\usepackage{hyperref}	% enlaces sin borde rojo y en negro
\hypersetup{ 
    colorlinks=true,
    allcolors= black
}


\usepackage{multicol}	% tablas de multiples columnas



\begin{document}
%\begin{center}
%  \textsc{\large Visualización | TP 1}\\
%\end{center}

\begin{center}
  \includegraphics*[width= 1.1\textwidth]{./top10_10s}
\end{center}

\newpage
\section*{¿Qué historia/s están intentando transmitir?}
\begin{itemize}
  \item Los incendios son más frecuentes en este siglo.
  \item Por tamaño de la provincia no son las más afectadas las patagónicas.
\end{itemize}


\section*{¿Qué aspectos de los datos permanecen ocultos?}
%(No es necesario visualizar todos los campos de los datos).
\begin{itemize}
  \item Si fueron intencionales. Los datos no lo detallan por cada provincia. Por tanto se trabajó con el total para poder comparar todas.
  \item La gráfica solo muestra las diez de mayor número por superficie en la última década reportada.
\end{itemize}


% Sean rigurosos en la descripción, proveyendo argumentos racionales para justificar las decisiones de diseño. Estas decisiones incluyen: la elección de la visualización, el tamaño, el color, la escala, y el resto de los atributos perceptuales, así como también el ordenamiento y las transformaciones de datos propuestas.
\section*{¿Cómo estas decisiones hicieron más efectiva la comunicación de los datos?}
\begin{itemize}
  \item Limité el número de provincias a diez para no saturar el gráfico y dar pie al argumento sobre la "no preponderancia de las patagónicas".
  \item Barras horizontales se adaptaban más a este número para hacer más visible la diferencia semi-cuantitativa (el número en sí no es lo relevante)
  \item Se ubicó el caso mayor abajo para alejarle de lo que a priori entendí era la mejor ubicación de la caja donde se transmite la idea de que se tratan de datos que responden a un período de una década para cada color.
  \item Los elementos de diseño, como tamaño relativo de los elementos del gráfico, escalas y colores son los que provee por defecto la biblioteca \emph{matplotlib} para su método que genera barras horizontales.
\end{itemize}


\section*{Herramientas utilizadas en el proceso}
\begin{description}
  \item[Superficies] 
    Son las oficiales publicadas por el Instituto Geográfico Nacional en su página web. Tras descargarle gracias a la biblioteca \emph{requests} se importó a un DataFrame de \emph{pandas} la tabla con el análisis sintáctico de HTML que provee \emph{Beautifulsoup}.
  \item[Aritmética de datos]
    El promediado por década y normalización por superficie se realizó haciendo uso de los métodos de la biblioteca \emph{pandas}.
  \item[Gráficación]
    Se utilizó la sugerencia de barras horizontales agrupadas por el nombre de índice que hace \emph{pandas}. Haciendo uso de las opciones de \emph{matplotlib} se ajustó el tamaño de la tipografía, títulos de la caja de etiquetas, ejes y el conjunto del gráfico.
\end{description}


\end{document}